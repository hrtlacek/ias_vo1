%!TEX root = ../main.tex
\section{Idee der LV}
 \frame{\sectionpage}

\begin{frame}{Ziele}
Das Ziel der LV ist es, in einer kollaborativen Arbeitsweise eine interaktive Installation umzusetzen.

\begin{itemize}
    \item \textcolor{white}{Aneignung von Wissen und Fertigkeiten in Theorie und Praxis}
    \item \textcolor{white}{Koordination von Expertise}
    \item \textcolor{red2}{Umsetzung einer Projektidee}
\end{itemize}
    
\end{frame}

\begin{frame}{Was für ein Projekt?}
Beispiele später, Grundzüge:

\begin{itemize}
    \item Interaktive Aspekte (daher Sensorik)
    \item Audio-Aspekte (Sonifikation, Sound-Design)
    \item Video-Aspekte (Visualisierung)
\end{itemize}

\end{frame}

% \begin{frame}{Blocks}
% \begin{block}{begin block}
% There's a block
% \end{block}

% \begin{alertblock}{begin alertblock}
%     there's a alert block 
% \end{alertblock}

% \begin{exampleblock}{begin example block}
% here comes example
% \end{exampleblock} 

% \end{frame}


% \begin{frame}{Blocks}
    
% \begin{theorem}
%     Here comes a theorem
% \end{theorem}

% \begin{proof}
%     Here comes the proof
% \end{proof}


    
% \end{frame}
