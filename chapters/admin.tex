%!TEX root = ../main.tex

\section{Administratives}
 \frame{\sectionpage}

\begin{frame}{Benotung}

 
\begin{table}[]
\begin{tabular}{l|l|l}
Pitch & Zwischenpräsentation & Projektabschluss \\ \hline
10\%  & 30\%                & 60\%       
\end{tabular}
\end{table}



\end{frame}


\begin{frame}{Notenschlüssel}

% 100-87: 1
% 86-75: 2
% 74-63: 3
%  62-50: 4
%  <50: 5

 
\begin{table}[]
\begin{tabular}{l|l}
Punkte & Note \\ \hline
0-50   & 5    \\
65-51  & 4    \\
80-66  & 3    \\
90-81  & 2    \\
100-91 & 1   
\end{tabular}
\end{table}

\end{frame}


\begin{frame}{Aufgaben und wichtige Daten}
% \[inline]{check dates}
\begin{table}[]
\begin{tabular}{l|l|l}
\emph{Was}             &  \emph{Wie}           & \emph{Wann}	\\ \hline
Gruppeneinteilung      & Online (eCampus)      & bis 11.\,3.   \\
% Zwischenprüfungen     & 5-10 min Ecampus      & Laufend	\\
Projekt-Pitch          & Informelles Gespräch  & 25./26.\,3.	\\
Zwischenpräsentationen & 10 min Referat        & 28./29.\,4.	\\
Projekt-Support        & Freies Arbeiten in UE & Mai/Juni	\\
Abschlusspräsentation  & 15 min Präsentation   & Prüfungswoche \\
Projektdokumentation   & Showreel-Eintrag      & Prüfungswoche
\end{tabular}
\end{table}

Abschlusspräsentation: Erstantritt erste Prüfungswoche, Zweitantritt zweite Prüfungswoche.

\end{frame}



\begin{frame}{Bewertungskriterien}
\begin{itemize}
\item \emph{Pitch}: Ungefähre Projektidee, Einschätzung der Arbeitspakete und Arbeitsteilung.
\item \emph{Zwischenpräsentation}: Vorhandene Ergebnisse, Einschätzung der verbleibenden Arbeitspakete und realistischer Ausblick auf den Projektabschluss.
\item \emph{Abschlusspräsentation}: Funktionierender Projektaufbau, Beschreibung der Funktionsweise (Gespräch).
\item \emph{Dokumentation}: Vorhandenheit. Keine Note ohne Doku auf Showreel.
\end{itemize}




\end{frame}
