%!TEX root = ../main.tex

\section{Administratives}
 \frame{\sectionpage}

\begin{frame}{Benotung}

 
\begin{table}[]
\begin{tabular}{l|l|l}
Pitch & Zwischenpresentation & End Projekt \\ \hline
10\%  & 30\%                & 60\%       
\end{tabular}
\end{table}



\end{frame}


\begin{frame}{Notenschlüssel}

% 100-87: 1
% 86-75: 2
% 74-63: 3
%  62-50: 4
%  <50: 5

 
\begin{table}[]
\begin{tabular}{l|l}
Punkte & Note \\ \hline
0-49   & 5    \\
62-50  & 4    \\
74-63  & 3    \\
86-75  & 2    \\
100-87 & 1   
\end{tabular}
\end{table}

\end{frame}


\begin{frame}{Tasks / Important Dates}
% \[inline]{check dates}
\begin{table}[]
\begin{tabular}{l|l|l}
\emph{Was}   &  \emph{Wie}             &	\emph{Wann}	\\ \hline
Gruppen Einteilung   & Am Ecampus              &		29.3. \\  
% Zwischenprüfungen        & 5-10 min Ecampus              &	Laufend	\\
Projekt Pitch        & Informelles Gespräch              &	7.4.	\\
Zwischenpresentationen        & 10 min Referat              &	23./24.5	\\
Project Support      & Frei Abeiten in UE              &	Mai/Juni	\\
Project Presentation & 15 min Presentation & Prüfungswoche \\
Project Doku + Submission & Showreel Eintrag & Prüfungswoche
\end{tabular}
\end{table}

Projekt Präsentation: Erstantritt erste Prüfungswoche, Zweitantritt zweite Prüfungswoche.

\end{frame}



\begin{frame}{Bewertungskriterien}
\begin{itemize}
\item \emph{Pitch}: Ungefähre Idee presentieren, einschätzung was ist zu tun. Arbeitsteilung.
\item \emph{Zw.Präsentation}: Vorhandene Ergebnisse, realistischer ausblick.
\item \emph{Endpresentation}: Funktionalität.
\item \emph{Dokumentation}: Vorhandenheit. Keine Note ohne Doku auf Showreel.
\end{itemize}




\end{frame}
